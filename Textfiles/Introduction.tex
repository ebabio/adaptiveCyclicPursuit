\section{Introduction}
Circular formations arise in natural behaviors in the animal kingdom: schools of fish \cite{parrish2002self} or groups of vultures stalking a prey exhibit this behavior according \cite{cone1962thermal}. From an aerospace point of view the vulture behavior is pretty remarkable since they use thermals to stay airborne with little or no effort. This behavior is also interesting for engineering applications. Using thermals to minimize fuel consumption for loitering aircraft is an interesting application \cite{akos2010thermal}. But we can think of other applications, such as UAVs being used for observation as proposed by Ma and Hovakimyan in \cite{ma2013cooperative}. 

In the simplest Cyclic Pursuit setup, each agent $i$ follows the immediately preceding agent denoted as $\mod(i+1,n)$ so the global connectivity graph is a strongly connected ring directed graph. In this condition all agents will achieve consensus simultaneously for random initial conditions as shown by Richardson \cite{richardson2001nonmutual}.

If instead of trying to capture the preceding agent we try to maintain a constant bearing with respect to it then circular formations can be achieved, which is the topic interest here. Some conditions for cyclic formations can be found are stated by Tabuada \cite{tabuada2001cylic}. The general solution for this strategy are spirals converging or diverging from the center and the circular formation is just the transition between them depending on some global conditions. Marshall and Francis \cite{marshall2004formations}  prove the single integrator to be stable using circulant matrices with is a single eigenvalue at the origin and all others are stable, a recurring result for distributed algorithms. 

Ramirez et al \cite{ramirez2010distributed} have extended the algorithm to control the radius of the formation and double integrators. The control of the radius relies on controlling the distance to the previous agent, if we know the relationship between the formation radius and the distance to the previous agent then the converging or diverging behaviors of the spiral can be excited. Some other notable work on single integrator dynamics is that of Zhao \cite{zhao2014distributed} where formations can be achieved using bearing-only measurements which is a new approach to the topic.

However, by choosing the non-holonomic steered particle our model is more restrictive and with a greater interest to the aerospace field. Typically, maneuverability on the longitudinal axis is restricted for fixed-wing aircraft where the range of accelerations is usually quite small and not all velocities are possible. 

The Cyclic Pursuit algorithm for non-holonomic particles were first introduced by Justh and Krishnaprasad in \cite{justh2002simple}, \cite{justh2003steering}. Marshall and Francis \cite{marshall2004formations} proved that the non-holonomic evenly-spaced  formation is locally stable by linearizing the formation about the equilibrium invariant set. Global conditions for convergence to the circle are given by Galloway, Justh and Krishnaprasad in a later work\cite{galloway2013symmetry} by the defining the dynamics of the Cyclic Pursuit manifold. They show that any initial state will converge to this manifold and then they study the behavior of the formation in terms of some global conditions. The global conditions for convergence to a circle are found irrespective of the spacing between agents and how those are related to the converging and diverging behavior.

Advancements on the robustness have been made by Fathian et al \cite{fathian2016distributed}. Their approach is based on defining some target position of the agent as a function of the preceding and trailing agents. An extra measurement is included but the algorithm becomes robust to changes in the the behavior of the network. Morbidi et al \cite{morbidi2010maintaining} have studied the effect of communication and sensor range in achieving the formation by focusing on the initial conditions and the transients before the formation is achieved. Sharma et al \cite{sharma2012cylic} have introduced a collision avoidance into the double integrator problem by means of a gravity-like repulsion force, and with agent with a modified control law that prevents the formation from diverging. Daingade et al \cite{daingage2016failsafe} have addressed the convergence with no cordination but relying on bearing measurements to all the agents in the network which is a somehow similar approach.

All these algorithms rely on local coordination, but still use some global information or coordination. A ring directed graph (di-graph) is needed and the conditions for convergence are given in terms of global characteristic that have to previously agreed. Pavone and Frazzoli \cite{pavone2007decentralized} define a strategy to achieve a ring di-graph by exploiting the fact that a circular formation will have all its agents in the boundaries of a convex set. An agent can find if it is in the convex set boundary of the formation and in the negative case move there. Once all agents are in the border they can follow the preceding agent that in this case will be in an extrema of the observed bearings to all other agents.

As of now we know that a circle can be achieved if some global conditions are met. But no work has addressed the fact the challenge of arriving to a circular formation with no global coordination at all for any possible network. Here lies the innovation of this work, we propose an adaptive local coordination control law that ensures the global conditions are met without any previous coordination or information. To the best knowledge of the authors, this is the first approach to achieve this feature. This is our main result, and it is proved to converge to a circle. These circular formations have no predetermined radius, or spacing. Two strategies for achieving a desired radius and spacing are proposed. The main result and the two proposed strategies are shown to converge to controlled circular formations in simulations for several cases including time-varying networks.

This text is organized as follows. Section 1 has defined the motivation for the problem. Section 2 introduces the Cyclic Pursuit algorithm as defined in \cite{galloway2013symmetry}, conditions for convergence and some issues that may arise. Section 3 introduces the main contribution, the adaptive control law, defining its assumptions, to allow for circular convergence with no global information. Section 4 introduces the complimentary results, control laws for the control of the radius and the spacing og the formation discussing their implications on measurements. Section 5 illustrates the results with simulation showing some issues of the initial algorithm and shows convergence for the adaptive algorithm under a variety of conditions. Section 6 provides some conclusions and discusses further work. Partial proofs for the convergence of the adaptive algorithm can be found in Annex A.