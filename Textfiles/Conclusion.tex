\section{Conclusion}

An adaptive control law is proposed that is able to make network converge to a circle using cyclic pursuit. The adaptive control law modifies the parameters used in cyclic pursuit in a distributed way in order to satisfy the global conditions for convergence. Furthermore, the adaptive control law uses only local measurements and does not need for any coordination between the agents. Beyond convergence to a circle, control laws for controlling the radius of the formation and achieving equal spacing are proposed. 

These three features: convergence to a circle, radius control and equal spacing (adding an extra measurement); allow for practical use of the algorithm for networks using only relative positioning. This is done by preserving the most relevant features of the original cyclic algorithm: global convergence and no need for communication between the agents. The proposed algorithm is not only feasible and implementable, it is also robust to changes in the network which is highly desirable for its integration into more complex systems.

Future work should address proving the convergence of the algorithm from a theoretical point of view. Only partial results for convergence are given in the Appendix. From a practical point of view, the integration of these findings with the strategy defined in \cite{pavone2007decentralized} for defining di-graphs in a decentralized way would allow to use the algorithm with no global coordination at all. Furthermore, it would be interesting to introduce control of the formation center when some agents have absolute positioning. Algorithms for circumnavigation such as those defined in \cite{deghat2010target} and \cite{deghat2014localization} could be of use for this purpose.